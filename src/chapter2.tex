\chapter{Tervezés, megvalósítás}

\section{A játék}

Mielőtt a tervezésnek nekiálltam volna, a feladatkiírás alapján megvizsgáltam a
piacon jelenleg elérhető, hasonló jellegű játékokat (pl. agar.io, slither.io), a
játék végleges szabályait ezek figyelembevételével alakítottam ki, illetve a
lehetőségekhez mérten próbáltam utána járni az általuk alkalmazott technikai
megoldásoknak is.  Itt a lehetőségeim rendkívül korlátozottak voltak, hiszen a
vizsgált szoftverek nem voltak nyílt forráskódúak.

\subsection{Specifikáció}

A játékban tetszőleges számú játékos vesz részt, ahol minden játékos egy űrhajót
irányít egy két dimenziós, potenciálisan végtelen nagy játéktéren. A játék célja
minél több pont összegyűjtése, amit másik játékosok űrhajójának
megsemmisítésével lehet szerezni, ehhez pedig az űrhajókon lévő fegyvereket
lehet használni.

A játék igyekszik a newtoni fizika szabályait betartani, azaz az űrhajókon lévő
hajtóművek forgatónyomatékot és erőt fejtenek ki az űrhajón. A játékos
alapvetően a navigációt a hajtóművek toló erejének szabályozásával tudják
végrehajtani, ez azonban meglepően nehéz. Így a játék lehetőséget ad egyedi
vezérlés beprogramozására, azaz a vezérlés scriptelhető. Az így készített
,,fedélzeti számítógép" minden adathoz hozzáfér, amihez maga a játékos is, így
lehetőség van az egyszerű navigációs szubrutinok mellett komplex műveletek
végrehajtására is, sőt akár teljesen autonóm működés is megvalósítható.

A fenti specifikáció mellett az alábbi megszorításokat alkalmaztam:

\begin{itemize}
  \item A játékosok száma maximalizálva van, hogy az ebből adódó esetleges
    teljesítményproblémákat megelőzzem. Hogy a játék mégse legyen ilyen
    korlátozott, a játék képes több \emph{szoba} elindítására is, így a szerver
    egy nagyon komplex játék helyett több kisebb létszámú szobát kezel.
  \item A játéktér mérete limitálva van, ezen túllépve a játékos először
    figyelmeztetést kap, majd az űrhajója megsemmisül. Ezzel a játék
    izgalmasabbá válik, hiszen a felhasználók könnyebben megtalálhatják egymást.
  \item A játékban elérhető sebesség korlátozva van, így nem lehet korlátlanul
    nagy sebességeket elérni. A fizikai modell alapvetően ezt nem tiltaná, de a
    korlátos játéktér miatt ez amúgy sem célravezető, így inkább mesterséges
    limittel előztem meg a problémát.
\end{itemize}

\section{Architektúra}

Az alkalmazás alapvetően két szoftverkomponensből áll, egy kliens oldali
JavaScript alapú Single Page Application-ből, illetve egy szintén JS alapú,
Node.JS felett futó szerver alkalmazásból.  A kliens egy minimális HTML kódba
van beágyazva, a böngészőbe ezt a fájlt betöltve (azaz a játék oldalára
ellátogatva) elindul az alkalmazás, automatikusan csatlakozik a szerverre,
készen áll a játékra.

\section{Szerver oldali megvalósítás}

A szerver alkalmazás egy \texttt{koa} alapú  HTTP webszerverből, és egy egyszerű
Websocket szerverből áll. A HTTP interfész fogadja az új játékosok
regisztrációit, illetve elvégzi az azonosításukat, másodlagosan pedig a böngésző
számára szükséges állományokat szolgálja ki. Erre csak fejlesztés közben van
szükség, \emph{production} környezetben ennek részleteit \aref{sec:nginx}.
fejezet tárgyalja.

\subsection{Adatbázis}

Node.JS környezetben különösen népszerűek a NoSQL megoldások, pl. a
MongoDB\cite{mongo}, hiszen képes közvetlenül JavaScript objektumokat tárolni,
azokban hatékonyan keresni. A játék implementációjához azonban én a hagyományos
relációs adatbázisoknál maradtam, mivel a tárolt adatok felépítése nem igényli a
komplex, több szintű objektumok tárolását.

A konkrét választás végül a PostgreSQL\cite{postgres} rendszerre esett.
Alternatíva lehetett volna a MySQL rendszer, azonban azzal korábbi munkáim során
már foglalkoztam, és mindenképpen szerettem volna megismerkedni az
alternatíváival.

\subsection{Autentikáció}

Egy olyan rendszernél, melyben szükség van autentikációra, különösen kritikus a
biztonságosság kérdése. A probléma megoldásánál figyelembe kell vennünk, hogy a
számítógépek számítása kapacitása még mindig exponenciális növekedést mutat
annak ellenére, hogy az órajelek már a szilícium-alapú chipek fizikai
korlátaiknál vannak\cite{moore}, azaz a gyorsulás már inkább az egyre több
párhuzamosan dolgozó számítási egységnek köszönhető.

A biztonságos adattároláshoz szükséges, hogy a felhasználók jelszavát ne
közvetlenül tároljuk, hanem olyan formában, hogy abból egy esetleges támadás
esetén semmiképpen ne lehessen visszafejteni az eredeti jelszót.

A transzformációval szemben támasztott követelmények:

\begin{itemize}
  \item Legyen \emph{hash} függvény, azaz tetszőleges méretű bemenetet fogadjon,
    és fix méretű kimenetet adjon.
  \item Ne legyen invertálható, azaz ne lehessen a $h(x)$ hash alapján az $x$
    értékét könnyedén megmondani.
  \item Két ugyanolyan jelszó esetén se legyen ugyanaz a kimenet, hiszen ez egy
    kiszivárgott jelszó esetén az összes többi ugyanolyan jelszót is leleplezné.
  \item Legyen kellően lassan számolható. Azaz paraméterként tudjuk állítani az
    elvárt számítási kapacitást is. Bár ezzel a feltétellel saját magunknak is
    plusz munkát okozunk, a feltételezésünk az, hogy a felhasználók jelszavas
    azonosítását viszonylag ritkán kell elvégezni, azaz itt a cél elsősorban egy
    esetleges támadás esetén a támadó dolgának megnehezítése.
\end{itemize}

A nem-invertálhatóságot a kriptográfiai hash függvények teljesítik, így
mindenképpen ezekből kell építkeznünk. A kimenetek különbözőségét egy
jelszavanként különböző, ún. \emph{salt} értékkel érhetjük el, amit az inputhoz
hozzáfűzünk a hashelés előtt. A salt értéke egy tetszőleges véletlenül generált
érték, a nem szükséges titkosítva tárolni.

Az utolsó pont megoldására (így a szerveren történő autentikációra) egy ún.
KDF-et (\emph{key derivation function}) használtam. Az ilyen függvények képesek
\emph{key streching}-re, azaz paraméterként megadhatunk egy iterációs számot,
amivel a függvény futását lassíthatjuk (a magasabb iterációs szám eredményezi a
lassabb futást).

Az elkészült alkalmazásban a \texttt{pbkdf2} algoritmust használtam, a szerveren
a felhasználók jelszavait pedig egy sorosított JSON objektumként tároltam, ami
minden szükséges paramétert tartalmaz a validációhoz. Így pl. az \emph{almafa}
jelszó egyik lehetséges tárolási formátuma az adatbázisban

\begin{js}
{
  "hash":"4bofZYTh+TIeprLIisByLaB7BvZ8FcmqvyRIyDrmmCWDT97fwexc...",
  "salt":"QqyMsCFXo5A9L2apbyzJaR3cxAcgPdTBg6olfZW0Q/XiTWOV++/c...",
  "iterations":100000,
  "len":128,
  "digest":"sha512"
}
\end{js}

Az iterációk száma így folyamatosan szabályozható, akár a regisztráció idejének
függvényében módosítható, illetve a jelszavak akár minden belépésnél
frissíthetők az adatbázisban, így azok mindig kellően nagy iterációs
paraméterrel fognak rendelkezni.

\subsection{Munkamenetkezelés}

A HTTP egy állapotmentes protokoll, így ha szükségünk van az állapot
karbantartására (az egyik legkézenfekvőbb ilyen tulajdonság pl. az aktuális
felhasználó azonosítója), akkor azt nekünk kell megoldanunk. A tipikus megoldás
a problémára, ha egy azonosítót rendelünk a felhasználó munkamenetéhez (pl.
bejelentkezéskor), ezt egy HTTP \emph{cookie}-ban eltároljuk, majd a szerveren
az azonosítóhoz tartozó munkamenet-objektumot beolvassuk, illetve azt
módosítjuk.

Ezzel szemben az én megoldásom nem igényli az adatok szerveroldali tárolását,
így az architektúra sokkal egyszerűbb. A munkamenetkezelés továbbra is
cookie-alapon történik, azonban az egész munkamenet objektum ebben a sütiben
van eltárolva.

Természetesen szeretnénk, ha ezt a felhasználó nem tudná manipulálni (pl. átírni
a saját felhasználó azonosítóját egy másik játékosére). Ehhez újfent a
kriptográfiához fordulunk. A felhasználó számára kiküldött HTTP cookie-t
aláírjuk (hozzáfűzünk) egy HMAC-et (\emph{hash-based message authentication
  code}), ami biztosítja az adatok integritását (azaz a felhasználó nem tudja a
tartalmát megváltoztatni), illetve biztosak lehetünk benne, hogy a sütit mi
állítottuk ki.

Az elkészült szoftverben a JWT (\emph{json web tokens}) implementációját
használtam, ez támogatja a fent leírt eljárást, illetve más egyebeket is, pl.
képes az adat titkosítására is, én azonban ezt nem tartottam fontosnak, mert a
munkamenet szenzitív adatokat nem kezel.

\begin{js}
{
  "user": {
    "id": 1
  }
}
\end{js}

A fenti JSON objektum (17 bájtos karakterláncként reprezentálható) leírható az
alábbi aláírt karakterlánccal is (104 bájt)

\begin{verbatim}
eyJhbGciOiJIUzI1NiIsInR5cCI6IkpXVCJ9.eyJ1c2VyIjp7ImlkIjoxfX0.
x--ftgJrZBR7ASBWv64OUOB59jaQ905rzlncCG6p7-Q
\end{verbatim}

\section{Kliens oldali megvalósítás}

A kliens egy Single Page Application, azaz a hagyományos HTML alapú oldalakkal
szemben, ahol minden egyes alkalmazásképernyő egy új HTTP kérés-válasz ciklus
jelent, itt erre pusztán egyszer van szükség, minden további interakciót a
futtatott JavaScript kód kezel le, annak a feladata a különböző grafikus
komponensek megjelenítése, illetve szerverrel történő kommunikáció.

\subsection{Felhasználói felület}

A felhasználói felület két nagy részre bontható: egy virtual dom alapú
hagyományos böngészős felületre, illetve egy WebGL-alapú játékfelületre.

Az előbbi felelős a komplett alkalmazás állapotmenedzsmentjéért, azaz kezeli a
különböző képernyők közti navigációkat, illetve a kezdőképernyőn történő
regisztrációs/beléptető felület megjelenítéséért.

A navigációhoz a History API-t\cite{history} használtam, mely képes a böngésző
címsávjában lévő URL-t módosítani az oldal újratöltése nélkül, illetve megadja a
lehetőséget, hogy a böngésző előre/vissza gombjaival is képesek legyünk a
navigációra, ezzel is növelve a felhasználói élményt.

A felület megvalósításához nem \aref{sec:react} fejezetben bemutatott React
könyvtárat, hanem az Inferno\cite{inferno}-t használtam, mely tulajdonképpen egy
React-kompatibilis API-t nyújt, azonban az implementációja jóval letisztultabb,
ezáltal gyorsabb\cite{bench}, ezt még a React fejlesztői is
elismerték\cite{reactrewrite}.

\subsection{Grafika megvalósítás}

A játék egy teljes képernyőt kitöltő WebGL vásznat használ, azaz teljes
képernyős üzemmódban a natív platformokhoz hasonló élményt próbál nyújtani, ezt
mutatja be \aref{fig:screenshot1}. ábra. A shaderek kezelésére a
\texttt{gl-shader} modult használtam, ez képes a GLSL kód alapján a WebGL
kontextus felkonfigurálására, azaz automatikusan beállítja a shaderek
paramétereit: a vertexek értelmezéséhez szükséges \texttt{attribute} változók,
illetve a \texttt{uniform} változók kötését.

A textúrák betöltése JavaScript segítségével történik, egyszerű \texttt{Image}
objektumok segítségével, amit \texttt{.texImage2D} metódus segítségével töltünk
be a WebGL kontextusba. Magukat a textúrákat az OpenGameArt ingyenes textúrái
között találtam (\texttt{http://opengameart.org/}).

A szöveges tartalom megjelenítése egy fájó pont volt számomra, a WebGL-ben ez
viszonylag körülményes, így kerülőmegoldásként a WebGL vászna fölé egy másikat
is helyeztem, amire a Canvas API segítségével rajzoltam ki szöveges tartalmat:
ennek segítségével jelenítettem meg az egyes játékosok pontjait.

\fig{width=0.6\textwidth}{screenshot-1.png}{Képernyőkép a WebGL-alapú
  grafikáról}{screenshot1}

\subsection{Játéklogika}

Mind a kliens, mind a szerver alkalmazás központi eleme a játékszabályok
betartásáért felelős üzleti logika. Ennek a feladata, hogy a hozzá beérkező JSON
sorosított utasításokat végrehajtsa. A konkrét implementációban ezt a szerepet
egy \texttt{GameRunner} objektum látja el. Az objektum tartalmazza egy aktuális
játék állapotát a \texttt{state} mezőjében, illetve rendelkezik egy
\texttt{handle(action: Action): void} metódussal.

Fontos, hogy a játék állapotát reprezentálható objektum \emph{immutable}, azaz
nem módosítható. Helyette az állapot megváltozása esetén kötelezően egy
\emph{új} állapotot kell létrehozni.  Ez biztosítja, hogy bármelyik időpontban
\emph{snapshotot} készíthessünk a játék állapotáról, ami a lokális predikció
működéséhez szükséges. A predikció részleteivel \aref{sec:prediction}. fejezet
foglalkozik.

A \texttt{GameRunner} mentes minden külső függőségtől, működéséhez fontos, hogy
a \emph{handle} metódus csak a bemenő paramétertől, illetve a saját belső
állapotától függjön, tehát ennek a két adatnak kell minden (az új állapot
létrehozásához) szükséges információt tartalmaznia. Mivel a játék nem használ
\emph{ütemező} üzeneteket, így minden végrehajtandó akció mellé csatolni kell a
hozzá tartozó időbélyeget is. Ezeknek az információknak a tudatában elő lehet
állítani a játék állapotát, amire már a paraméterben kapott művelet is hatással
van. A végrehajtás két lépésben történik:

\begin{enumerate}
  \item Az akció időbélyege, és az aktuális időbélyeg alapján végrehajtjuk a
    fizikai szimulációt, azaz a különböző objektumokat adott időegységekkel
    léptetjük, amíg el nem érjük a kívánt időpontot. Itt történik meg minden
    determinisztikus játéklogika, azaz az ütközéskezelés, pontok számolása is.
  \item Végrehajtjuk a tényleges akciót (pl. új lövedéket helyezünk el a
    játéktéren a lövés akció hatására)
\end{enumerate}

A \texttt{GameRunner} mivel nem rendelkezik külső függőségekkel, így mind
kliens, mind szerveroldalon futhat. A kliens és a szerver alkalmazások csak
minimálisan kommunikálnak egymással, így mind a kliensen, mint a szerveren
szükséges az önálló játékpéldány futása. A lehetséges üzenettípusokkal
\aref{sec:messages}. fejezet foglalkozik.

\subsection{A játék objektumgráfja}\label{sec:gamestate}

A játék állapota egy hagyományos JavaScript \texttt{Object}, így bármikor JSON
sorosítható (és persze parse-olható), a tényleges struktúrája a FlowType
segítségével jól leírható (\texttt{State} típus):

\begin{js}
type Vec = [number, number];

type Thruster = {
  pos: Vec, ori: number, thrust: number,
};

type ShipType = {
  hull: number,
  thrusters: Thruster[],
};

type Ship = {
  id: number,
  type: number,
  owner: number,
  pos: Vec, vel: Vec, acc: Vec,
  ori: number, rot: number, tor: number,
  health: number,
  thrusters: number[],
};

type Projectile = {
  owner: number,
  pos: Vec, vel: Vec, ori: number, t: number,
};

type Player = {
  name: string,
  id: number,
  score: number,
};

type State = {
  ships: Ship[],
  projectiles: Projectile[],
  players: Player[],
  t: number,
  dt: number,
  ids: number,
};
\end{js}

A játékállapot tartalmazza az aktuális játékosok listáját: a játékosok egy
névvel, egy pontszámmal, és egy belső azonosítóval rendelkeznek, melyet a
\texttt{State::ids} változó segítségével generál a játék (egyszerűen egy növekvő
számsor szerint osztjuk ki az azonosítókat).

Az egyes lövedékeket szintén el kell tárolnunk az állapotban, ehhez szükséges
azoknak a pozíciója, iránya, illetve sebessége. Mivel nem szeretnénk, ha a
játékos saját magát is meg tudja sebesíteni, így tároljuk a tulajdonos játékos
belső azonosítóját. A másik indok a tulajdonos tárolására, hogy az űrhajók
megsemmisülését okozó lövedék gazdáját ismernünk kell a pontok számolásához.

Végül szükség van az egyes űrhajók adataira is, ez a fizikai jellemzőket
(pozíció, sebesség, gyorsulás, orientáció, forgási sebesség, forgatónyomaték),
illetve a játékszabályok szempontjából fontos értékeket (a hajó pajzsának
állapota, illetve a tulajdonos játékos azonosítója) tarkarja. A navigációhoz el
kell tárolni az egyes hajtóművek állapotát is, ez minden hajtóműhöz egy $0$ és
$1$ közötti erősséget jelent. Ahhoz, hogy ezeket az értékeket értelmezzük,
szükség van a hajó típusára is, ez a típusának az indexét jelenti, amit a játék
állapotán kívül tárolunk, globális értéknek tekintjük. Ez a típus mutatja meg
hajó maximális védelmét, illetve adja meg az egyes hajtóművek leírását, azaz a
pozíciójukat, irányukat a hajó középpontjához képest, illetve a maximális
tolóerejüket.

\section{Kliens-szerver kommunikáció}

\subsection{Websocket}

Az egyik célja a dolgozatnak a minél optimálisabb Websocket kommunikáció volt,
azaz meg kellett keresnem azt a minimális mennyiségű üzenetet, ami feltétlenül
szükséges a kliensek (és a szerver) közti szinkronizációhoz.

Ha feltételezzük, hogy a rendszer determinisztikus, akkor azonos kezdőállapotok
mellett (és külső események nélkül) nincs szükség egyáltalán szinkronizációra
(hiszen mindegyik példány ugyanazokon az állapotokon megy keresztül a
determinisztikus működés miatt).
Könnyen beláthatjuk, hogy a szinkronizáció pont ezek miatt a külső események
miatt szükséges. Ezek az események a játékosoktól (vagy ágensektől) származó
inputok, illetve a szerver által generált események, tehát a szinkronizációhoz
elegendő ezeket az eseményeket eljuttatni mindenkihez.

A módszer másik előnye, hogy ezeket az eseményeket felvéve (azoknak az
időzítésével együtt) rekonstruálni tudjuk a rendszer állapotát tetszőleges
időpillanatban, így pl. az egyes játékok visszanézhetők utólag. Ez a funkció
azonban sajnos az idő hiánya miatt nem valósult már meg.

\subsection{Üzenetek}\label{sec:messages}

A Websocket üzenetek JSON sorosított objektumok, minden üzenetnek van
\emph{típusa}, illetve valamilyen \emph{payload}-ja, ez tartalmazza az üzenethez
esetlegesen tartozó kiegészítő információkat. Az üzenet típusa felsorolt típus
(\emph{enum}), ezt azonban a JSON szintaktika (illetve a JavaScript nyelv) nem
támogatja, így kerülőmegoldásként azokat stringekként reprezentáltam. Ekkor
persze fennáll a veszélye annak, hogy a kliens és a szerver definíciói
különböznek, ennek elkerülése érdekében egy közös definíciós fájlt használtam,
ebben vannak definiálva a lehetséges üzenettípusok, így biztosak lehetünk benne,
hogy a két fél azonos szótárral dolgozik.

A módszer előnye, hogy a reprezentációt bármikor módosítani lehet, így adott
esetben kompaktabb reprezentációt is lehet nekik adni (pl. más nyelvek belső
reprezentációjához hasonlóan egyszerű számokkal helyettesíteni a típust), ezzel
is csökkentve az üzenetek által igényelt sávszélességet. Fejlesztés közben
azonban fontosabbnak éreztem, hogy az üzenetek ránézésre is követhetőek
legyenek, így ott olvasható, szöveges reprezentációt használtam.

Az ütemezés közvetlenül a végrehajtott akciókon keresztül történik, így szükség
van az időbélyegek eltárolására is, ez a játék kezdete óta eltelt időt jelenti.

\subsubsection{Űrhajóvezérlés}

Az űrhajók navigációját közvetlenül a hajtóművek szabályozásával lehet
megoldani, így az üzenetben csak meg kell címezni a megfelelő hajtóművet,
illetve a kívánt tolóerőt egy normalizált 0-1 skálán. A címzés a hajó
azonosítója, illetve a hajtómű indexe alapján történik.

\begin{js}
  {
    type: 'game:thrust',
    id: number,
    index: number,
    strength: number,
    t: number,
  }
\end{js}

\subsubsection{Fegyvervezérlés}

A fegyverek hasonló elven működnek, elegendő pusztán a megfelelő hajó megcímzése.

\begin{js}
  {
    type: 'game:shoot',
    id: number,
    t: number,
  }
\end{js}

\subsubsection{Játékos hozzáadása}

A játékos hozzáadásához elegendő a játékos nevét megadni, ez egyedien azonosítja
a játékost.

\begin{js}
  {
    type: 'game:add-player',
    name: strig,
    t: number,
  }
\end{js}

\subsubsection{Játékos eltávolítása}

Ha a játékos elhagyja a játékot, arról értesülnie kell a többi játékosnak is,
erre szolgál ez az üzenet, a címzése a játékos azonosítója alapján történik.

\begin{js}
  {
    type: 'game:remove-player',
    id: number,
    t: number,
  }
\end{js}

\subsubsection{Űrhajó hozzáadása}

A játéktérhez egy új űrhajót adhatunk hozzá, ehhez szükség van a tulajdonos
játékosra (az azonosítójára), az űrhajó típusára, illetve az űrhajó
kezdőállapotára, pl. a pozíciójára, illetve orientációjára.

\begin{js}
  {
    type: 'game:add-ship',
    ship: number,
    owner: number,
    pos: Vec,
    ori: number,
    t: number,
  }
\end{js}

A játékos eltávolításával ellentétben az űrhajók feltakarítása automatikus, ha
egy űrhajó gazdátlanná válik, automatikusan eltűnik.

\subsection{Lokális predikció}\label{sec:prediction}

A játék konzisztenciájáért a szerver a felelős, azaz a klienseknek a szerver
állapotát kell replikálniuk. Azonban a két fél közötti kommunikáció késleltetése
nem elhanyagolható, így ha szigorúan a szerver állapotát replikálnánk, az
érehető ,,lag"-ot okozna a játékban, azaz a felhasználó akár több száz
milliszekundumot is kénytelen lenne várni, mire a játék reagálna a leütött
billentyűire.

A fenti probléma elfedésére a kliens működését úgy módosítottam, hogy az képes
legyen \emph{lokális predikcióra}, azaz a kliens nem várja meg a szerver
válaszát, hanem automatikusan sikeresnek tekinti azt. Ezt a modellt mutatja be
sematikusan \aref{fig:prediction}. ábra.

\fig{width=\textwidth}{prediction.eps}{A lokális predikció modellje}{prediction}

A kliensek mindig rendelkeznek a szerver legutolsó ismert állapotával, ez a
\texttt{snapshot}, illetve van egy folyamatosan változó lokális állapotuk is, ez
a \texttt{state}. Ezenk kívül nyilvántartjuk a játékos által végrehajtott, ám
szerver által még nem visszaigazolt akciókat, az akció keletkezésének
időpontjával együtt (\texttt{queue}). A predikció működéséhez szükséges két
lényeges pont:

\begin{itemize}
  \item Minden új akció létrejöttekor hozzáfűzzük azt a \texttt{queue}-hoz,
    illetve végrehajtjuk azt a lokális állapoton is (\texttt{state}).
  \item Minden szervertől érkező esemény hatására frissítjük a
    \texttt{snapshotot}, a \texttt{queue}-ból eldobjuk azokat az eseményeket,
    amelyek időbélyege az érkezett eseménynél kisebb. Ezt azért tehetjük meg,
    mert a WebSocket az üzenetek sorrendjét nem változtatja meg, azaz ha egy
    későbbi eseményről kapunk visszaigazolást, akkor egy korábbi eseményről már
    biztosan nem fogunk.
  \item A \texttt{queue} megmaradt elemeit sorban végrehajtjuk az új
    \texttt{snapshot}-on, ezzel megkapjuk az új lokális \texttt{state}-et.
\end{itemize}

\section{Fizikai szimulációs modell}

A feladatban kitűzött egyik sarokpont a minél pontosabb fizikai modell
használata volt, így igyekeztem olyan kompromisszumos megoldást találni, amely
kellően pontos, ám figyelembe veszi, hogy a szimulációt böngészőben, limitált
erőforrásokkal kell futtatni.

A szimulációnak folyamatosnak kell lennie, ez optimális esetben 60 képkockát
jelent másodpercenként, azaz az egész szimulációnak kevesebb, mint 17
ezredmásodperc alatt le kell futnia.  Ha figyelembe vesszük, hogy itt akár
egyszerre több száz objektum viselkedését kell számon tartanunk, akkor érdemes
nagyon okosan megválasztanunk a szimulációs modellünket.

Az űrhajókat a két dimenziós síkon lévő merev kiterjedt testekkel modelleztem, a
testek mozgását (az elmozdulását illetve orientációját) pedig a rá ható
tetszőleges $n$ számú erőkkel írtam le. Ezekből az $F_i$ erőkből kiszámolható a
testre ható $F$ eredő erő, illetve $M$ forgatónyomaték.

\begin{align*}
F &= \sum_{i=0}^{n} F_i \\
M &= \sum_{i=0}^{n} F_i \times r_i
\end{align*}

Ahol $r_i$ az $i$. erőkart (a test tömegközéppontjából az erő támadáspontjába
mutató vektor) jelenti.

Ebből a két mennyiségből képesek vagyunk kiszámolni idő szerinti integrálással a
test sebességének, illetve szögsebességének megváltozását, amiből szintén egy
újabb integrálással megkapjuk magát az $r$ helyvektort, illetve a $\theta$
orientációt a $t$ időpillanatban.

\begin{align*}
v(t) &= \int_{0}^{t} \frac{F(t)}{m} dt \\
\omega(t) &= \int_{0}^{t} \frac{M(t)}{I} dt \\
r(t) &= \int_{0}^{t} v(t) dt \\
\theta(t) &= \int_{0}^{t} \omega(t) dt
\end{align*}

Ahol $m$ a test tömege, $I$ pedig a test tehetetlenségi nyomatéka.  Az
egyenletben szereplő $F(t)$ és $M(t)$ nem biztos, hogy könnyen integrálható
függvények, így $r(t)$ és $\theta(t)$ csak közelítő módszerekkel számolható,
hiszen a rendelkezésre álló néhány ezredmásodperc egészen biztosan nem elegendő,
hogy szimbolikusan oldjuk meg a feladatot (ha egyáltalán meg lehet).

Ha pl. feltételezünk egy gravitációs mezőt, ott a testre ható erő függ a test
pozíciójától is, azaz egy másodrendű differenciálegyenletet kapunk.

\subsection{Euler módszer}

Az Euler módszer pontosan a fenti integrálási problémára nyújt egy közelítő
megoldást. A módszer lényege, hogy az ismeretlen függvényt (pl. a $v(t)$
sebességfüggvényt) a deriváltja segítségével közelítjük (ez a példánk esetében az
ismert $F(t)$ függvény).

\[
  v(t + \varepsilon) \approx v(t) + \varepsilon v'(t)
\]

Azaz a sebesség változását lineárisnak tekintjük az eltelt $\varepsilon$ idő
alatt. Ha az $\varepsilon$ kellően kicsi, akkor ez a közelítés valóban megállja
a helyét, a módszer hibája egyenesen arányos $\varepsilon$-nal.

Érdemes megjegyezni, hogy ha a hajóra ható erő és forgatónyomaték konstans,
akkor az elkövetett hibánk $0$, így a sebességre és a szögsebességre pontos
megoldásokat kapunk.  Természetesen az így kapott függvények már nem konstansok,
így az újabb integrálás már hibával fog járni.

Mivel a szimulációnak folyamatosnak kell lennie, így az $\varepsilon$ változót is
kicsinek kell megválasztanunk, azaz az elkövetett hibánk kellően kicsi lesz ahhoz,
hogy az ne legyen zavaró a játékos számára.

\section{Programozott vezérlés}

Az űrhajó vezérlése billentyűzetről történik, a \texttt{W} és \texttt{S}
billentyűkkel a sebességet lehet befolyásolni (mivel a vezérlet hajtóművek által
kifejtett erők vektora átmegy a tömegközépponton, így forgatónyomatékot nem
fejtenek ki), az \texttt{A} és \texttt{D} billentyűkkel a forgást lehet
szabályozni (ezek pedig csak forgatónyomatékot fejtenek ki). A fegyver a
\texttt{space} billentyűvel süthető el.

Az irányítás közel sem intuitív, hiszen az energiamegmaradás miatt az űrhajó
mindaddig mozgásban marad (illetve forog), amíg azt egy ellenkező irányú erővel
ellen nem kompenzáljuk. Ezek az általános navigációs manőverek sok gyakorlást
igényelnek, azonban nagyon könnyen automatizálhatóak, így jött a programozható
vezérlés bevezetésének gondolata.

Mivel az alkalmazás a böngészőben fut, így ezt a vezérlést is JavaScript
segítségével valósíthatjuk meg, a játékkal való interakciót egy egyszerű API-n
keresztül valósítottam meg:

\begin{itemize}
  \item A \texttt{GameClient} osztály egy-egy példánya egy-egy játékos
    reprezentációja, ez ajánl ki néhány használható metódust
    (\texttt{.thrust(index, strength)} illetve \texttt{.shoot()}), ezek
    segítségével képes a játékos irányítani az űrhajóját.
  \item A játék aktuális állapota (az egyes űrhajók helyzete, stb.) bármikor
    elérhető \aref{sec:gamestate}. fejezetben ismertetett struktúrán keresztül.
\end{itemize}

A játékos a vezérlés kódját egy szöveges mezőben szerkesztheti, ahol teljes
szabadságot kap. Annak érdekében, hogy a különböző programozási hibákból eredő
problémákat elkerüljük, az itt bevitt kód egy \emph{sandbox} környezetben, egy
WebWorker szálon fut. Így az esetleges hibás (vagy rosszindulatú) vezérlések
miatt keletkezett pl. végtelen ciklusból is van esélyünk kilépni (új vezérlést
feltölteni), erre egyébként nem lenne esélyünk, hiszen a saját JavaScript kódunk
ugyanazon a szálon futna, mint maga a játék, azaz egy végtelen ciklus magát a
játékot (a felhasználói felületet) is blokkolná.

A WebWorkerrel történő kommunikáció el van rejtve, ezzel a játékosnak nem kell
törődnie, veheti úgy, hogy a számára elérhető \texttt{GameClient} objektum
ténylegesen végzi a vezérlést (valójában persze a felhasználó kódja csak egy
proxy objektummal kommunikál, ami továbbítja a megjelenítő szálon futó valódi
\texttt{GameClient} objektumnak az utasításokat).

A vezérlés kódjára nincs megkötés, azonban érdemes az alábbi struktúrát követni:

\begin{js}
setInterval(() => {
  runCustomController(client, state);
}, 100);

function runCustomController(client, state) {
  // egyéni logika, ahol lehetőség van a client.thrust()
  // és client.shoot() metódusok meghívására.
}
\end{js}

A fenti kód $100$ milliszekundomonként meghívja az általunk írt vezérlőt, ahol
tetszőleges logikát valósíthatunk meg.

\section{PID-vezérlős célzás}

A diplomaterv egyik célja egy alapszintű mesterséges intelligencia elkészítése
volt, melyhez a imént felvázolt scriptelési lehetőségeket használtam fel.

A tervezési szakaszban ambiciózus terveim voltak erre vonatkozólag, azonban mint
azt tapasztaltam, egy teljesen autonóm vezérlő implementálása túl bonyolultnak
bizonyult, így végül egy jóval egyszerűbb, de ,,okos" vezérlőt készítettem el.

A vezérlő feladata, hogy a hozzá legközelebb eső űrhajót megkeresse, és
megsemmisítse azt, azaz a hajó orientációját szabályozva a megfelelő
időpillanatban automatikusan elsüsse a hajóra szerelt löveget.

Az orientáció szabályozásához egy PID vezérlőt használtam, ehhez a \\
\texttt{node-pid-controller} npm modult használtam fel. A program az orientáció
hibája alapján vezérli a forgatásért felelős hajtóműveket, majd az űrhajó
megfelelő orientációja esetén automatikusan elsüti a rászerelt fegyvert.

\section{Tesztelés}

A dinamikusan típusos nyelvekben a tesztelés kiemelten nagy szerepet játszik,
hiszen minden szintaktikailag helyes program futtatható, azonban ennek az ára a
futás idejű hibák megnövekedett száma.

Mivel az alkalmazás erősen támaszkodik a FlowType-ra, így számos programhiba már
fordítási időben kiderült, a mainstream nyelvekben megszokott
típusellenőrzéseken felül a FlowType a \texttt{null} értékekből eredő hibákat is
képes kiszűrni (erre a Java vagy C\# nyelvekben nincs igazán lehetőségünk), így
a tesztelés során lehetőségem nyílt rá, hogy ténylegesen az implementált
algoritmusok helyességét ellenőrző tesztekre koncentráljak.

A választásom \aref{sec:mocha}. fejezetben bemutatott Mocha keretrendszerre
esett, a tesztelés fókuszát pedig a játéklogikára helyeztem. Itt hozta meg a
gyümölcsét a funkcionális programozásra építő architektúra, hiszen elegendő volt
minden esetben a tesztelt metódusok visszatérési értékét tesztelnem, nem volt
szükség bonyolult \emph{mock}, \emph{dummy} és egyéb tesztelési segédobjektumok
létrehozására. Szintén kifizetődőnek bizonyult, hogy az adatstruktúráim
hagyományos JavaScript objektumok voltak, így a egységtesztelés során nem volt
szükség bonyolult inicializálókra, minden paramétert képes voltam objektum
literálokkal definiálni.

\subsection{Continous Integration}\label{sec:ci}

A tesztelés akkor igazán hasznos, ha fejlesztési folyamat szerves részét képezi.
A \emph{Continous Integration} (CI) módszer célja pontosan ez, a fejlesztés során
folyamatosan biztosítja, hogy a verziókezelő rendszerben keletkezett minden
változáskor automatikusan lefutnak a szoftverhez írt tesztek.

Az alkalmazás kódjának menedzseléséhez a \texttt{git} verziókezelő
rendszert\cite{git} használtam, illetve a kódot a \emph{GitHub}\cite{github}
által biztosított távoli repository-ban tároltam. A GitHub számos külső CI
szolgáltatással képes együttműködni, ilyen pl. a \emph{Travis CI}\cite{travis},
illetve a \emph{CircleCI}\cite{circle}. Én a \emph{Travis CI} mellett döntöttem,
leginkább a korábbi tapasztalataim alapján.

A szolgáltatás minden GitHubra felküldött kódváltozásról értesül egy webhookon
keresztül, melynek hatására lefutnak az általunk specifikált tesztek.  A Travis
egy Linux konténert, vagy virtuális gépet biztosít számunkra attól függően, hogy
szükségünk van-e rendszergazdai jogosultságokra a tesztek lefuttatásához.  A
tesztkörnyezetet illetve magukat a lefuttatandó a teszteket egy \emph{YAML}
leírófájlban adhatjuk meg, melyet az alkalmazás repository-ban kell elhelyeznünk
(\texttt{.travis.yml} néven). Itt kell beállítanunk a kívánt környezetet,
esetünkben ez a \texttt{node\_js}-t jelenti, illetve megadhatjuk, hogy milyen
Node.JS verziókon szeretnénk lefuttatni a teszteket.  A rendszer automatikusan
bekonfigurálja a futtatókörnyezetet (elérhetővé téve a megfelelő Node.JS
verziót), majd lefuttatja a teszteket, alapesetben a verziókat szekvenciálisan
futtatva (a szolgáltatás ezen része ingyenes, ha párhuzamosan szeretnénk több
verziót tesztelni, a fizetős verziót kell használnunk).

Mivel a fejlesztési, illetve az éles futtatókörnyezet ismert (7-es főverziójú
Node.JS), így a teszteket csak a \emph{latest} jelzővel ellátott Node.JS
verzióval futtattam le.

A rendszer képes a beállított környezet alapján automatikusan kitalálni, hogy a
program függőségeit hogyan lehet feltelepíteni, illetve a teszteket hogyan lehet
lefuttatni. Ez \texttt{node\_js} esetén az \texttt{npm install} illetve
\texttt{npm test} parancsokat jelenti. Természetesen lehetőségünk van ezeket a
beállításokat a \texttt{.travis.yml} fájlban felülírni, azonban erre nekem volt
szükségem a fejlesztés során.

A lefuttatott tesztek eredménye alapján (a teszt sikeres, ha a teszet futtató
parancs $0$ értékkel tért vissza, ami hagyományosan UNIX környezetben a sikeres
lefutást jelenti\cite{bash}) egy újabb webhookon keresztül a Travis értesíti a
GitHub rendszerét, ahol vizuális visszajelzést is kapunk a teszt eredményéről.

A szolgáltatás a gyakorlatban rendkívül hasznosnak bizonyult, hiszen nem
,,felejti el" lefuttatni a teszteket, ami manuális futtatásnál valószínűleg
általános probléma lenne, illetve szisztematikusan képes több környezetben is
tesztelni, ami kézi módszerekkel időigényes, repetitív folyamat. Fontos, hogy az
általa futtatott tesztek reprodukálhatóak, azaz nem feltételezik a környezet egy
adott állapotát, így számos olyan hibára is fényt deríthetnek, ami helyi
teszteléskor nem feltétlenül kerül elő (pl. egy függőség feltelepítése anélkül,
hogy az a \texttt{package.json}-ba is bekerülne).

\section{Üzemeltetés}

A szoftverfejlesztés folyamatának szerves része az elkészült szoftver
üzemeltetése is, ezt az alkalmazást fejlesztő mérnökök nagyon gyakran
hajlamosak elhanyagolni, mondván, hogy ez nem az ő feladatuk, hanem az
üzemeltető rendszergazdáké.  Az agilis szoftverfejlesztés elterjedésével
megjelent az igény a két terület - fejlesztés és üzemeltetés - közti szakadék -
ha nem is betemetésére - de annak a szűkítésére.  Így alakult ki a
\emph{DevOps} (a \emph{development} és \emph{operations} szavakból) kultúra,
melynek célja a fejlesztők és az üzemeltetők közti szorosabb együttműködés
támogatása.

\Aref{sec:ci}. fejezetben bemutatott Continous Integration rendszer mellett két
szoftvert használtam az elkészült alkalmazás beüzemelésére, az \emph{nginx}-et
illetve az \emph{Ansible}-t. Segítségükkel az alkalmazás üzembe helyezése gyors,
és ami a legfontosabb: megismételhető, azaz szinte bármilyen, a népszerű VPS
szolgáltatók által kínált virtuális (vagy akár fizika) gépre feltelepíthető.

Az elkészült játékot a Digital Ocean által biztosított virtuális gépen helyeztem
üzembe, illetve a játékot a \texttt{https://spacegame.danyi.me} címen tettem
elérhetővé.

\subsection{nginx}\label{sec:nginx}

Az \emph{nginx}\cite{nginx} egy általános célú \emph{reverse proxy}, illetve
webszerver.  Éles környezetben ez látja el a statikus fájlok kiszolgálásának
feladatát. Habár a Node.JS alapú szerver is képes erre, az \emph{nginx} jóval
kiforrottabb megoldás, számos olyan alapvető funkciót támogat, melyet a Node.JS
alapú társa nem (vagy adott esetben csak különböző kiegészítő modulokkal), ilyen
pl. a megfelelő HTTP \emph{Cache} fejlécek kezelése, illetve a HTTP kérés-válasz
párok alatt meglévő TCP kapcsolat optimális kezelése (nem nyit új kapcsolatot,
csak ha feltétlenül szükséges).

Másik előnye, hogy a szerver összeomlása esetén is válaszképes marad a szerver,
a hibáról legalább egy HTTP 502 hibaüzenet formájában kapunk információt.

Végül ez a szerver felelős a HTTP illetve WebSocket kapcsolat titkosításáért,
melyhez a szükséges tanúsítványokat a \emph{Let's encrypt}\cite{letsencrypt}
szolgáltatás segítségével szereztem be.

Az \emph{nginx} szerepét \aref{fig:nginx}. ábra mutatja be.

\fig{width=0.7\textwidth}{nginx.eps}{Az nginx szerepe az éles üzemben}{nginx}

A szerver alapvetően három feladatot lát el: kiszolgálja a statikus fájlokat, illetve
\emph{reverse proxy}-ként működik a Node.JS alkalmazás HTTP és Websocket
interfésze előtt. Ezt a három feladatot három egyszerű direktívával tudjuk
megadni:

\begin{nginx}
# HTTP reverse proxy a Node.JS szerverhez
location / {
  proxy_pass http://spacegame;
}

# Websocket reverse proxy a Node.JS szerverhez
location /connect {
  proxy_pass http://spacegame;
  proxy_http_version 1.1;
  proxy_set_header Upgrade $http_upgrade;
  proxy_set_header Connection "upgrade";
}

# Statikus fájlok kiszolgálása
location ~* \.(js|map|jpg|css)$ {
  root /home/picard/spacegame/assets;
}
\end{nginx}

\subsection{ansible}\label{sec:ansible}

Az szoftver üzembe helyezéséhez számos szoftver megléte szükséges, pl. szükség
van a megfelelő verziójú Node.JS futtatókörnyezet, szükség van minden használt
library feltelepítésére, stb. Ezek a műveletek repetitívek, így célszerű őket
automatizálni.

A probléma megoldására az \emph{Ansible}-t\cite{ansible} használtam, mely egy
általános célú IT automatizációs eszköz, az elvégzendő feladatokat egy egyszerű
\emph{yml} leírófájl írja le, a szoftver pedig képes azokat egy távoli gépen,
SSH kapcsolaton keresztül végrehajtani.
