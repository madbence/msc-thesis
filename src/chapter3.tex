\chapter{Tapasztalatok, továbbfejlesztési lehetőségek}

\section{Tapasztalatok}

A fejlesztés során a legnagyobb technikai problémát a kliens és a szerver közti
kapcsolat és szinkronizáció tesztelése jelentette, az egy-egy bug előidézéséhez
szükséges körülmények reprodukálása nagyon nehézkes volt, gyakran csak a
szerencsén múlott, hiszen a hálózat sebessége fölött csak minimális kontrollunk
lehet fejlesztés közben. Végül sikerült kihasználnom a tényt, hogy a szerver
kódja (egy része) képes a böngészőben is futni, így egy egyszerű proxy osztály
elkészítésével (ami a WebSocket kapcsolatot volt hivatott helyettesíteni) képes
voltam a teljes szinkronizációs logika tesztelésére a böngészőben, ami nagyban
megkönnyítette a manuális tesztelést, illetve a hibák debugolását.

\subsection{JavaScript fejlesztői eszközök}

Az elmúlt években a JavaScript alkalmazások fejlesztéséhez szükséges eszközök
komplexitása érezhetően megnőtt, ezt sokan negatívumként hozzák fel, azonban a
tapasztalataim alapján ez inkább előnyére vált a fejlesztési folyamatnak. A
\emph{Babel} segítségével képes voltam túllépni a böngészők változatos
JavaScript futtatókörnyezetei által szabott korlátokon, képes voltam az ES2015
nyújtotta előnyöket kihasználni, ezáltal sokkal olvashatóbb, logikusabban
struktúrált kódot írni.

A FlowType a fejlesztés során magabiztosságot nyújtott, minden egyes módosítást
bátran végeztem el, korábban ez közel sem volt így, típusellenőrzések nélkül a
paraméterek illetve a függvények visszatérési értékeiben bekövetkezett
változások egészen komplex, és nehezen visszakövetkező hibákat okoztak. A
FlowType-pal szemben gyakran felhozott egyik negatívum az annotációk okozta
rugalmatlanság, illetve hogy gyakran magának a metódus szignatúrájának
megfogalmazása nem egyszerű. Én úgy gondolom, hogy az okozott kellemetlenségek
messze eltörpülnek a típusosság által biztosított előnyök mellett.

A FlowType természetesen még egy új technológia, nem mentes a ,,furcsaságoktól",
az olyan helyzetektől, amikor az egyébként helyes kódot a rendszer túl szigorúan
ellenőriz, és helytelennek gondol:

\begin{js}
function foo(x: { y: ?string }): string {
  if (x.y) {
    bar();
    return x.y; // error
  } else {
    return 'baz';
  }
}
\end{js}

A fenti példában a \texttt{bar()} hívás elvben módosíthatja \texttt{x.y} értékét
(akár \texttt{null} vagy \texttt{undefined}-re is állíthatja), így a fordító
hibásnak ítéli meg a kódot, hiszen nem garantált a \texttt{string} visszatérési
érték.

\section{Továbbfejlesztési lehetőségek}

A fejlesztés során számos olyan új lehetőség jött elő, melyre a tervezés során
nem is gondoltam, ezek közül emeltem ki néhányat, melyek mind magát a játék
minőségét emelnék, illetve az üzemeltetést egyszerűsítenék le a jövőben.

\subsection{Moduláris vezérlés}

A programozott vezérlés teljes szabadságot ad a felhasználó kezébe, hiszen a
vezérlés teljes kódjának elkészítése az ő felelőssége. Ehhez azonban komplex
programozási ismeretek szükségesek, így ezt a funkciót csak a felhasználók
töredéke képes kihasználni.

Egy érdekes fejlesztés lehetne, ha a programozási ismeretekkel nem rendelkező
felhasználók is elérhetnék ezeket a funkciókat, valamilyen könnyen kezelhető
grafikus felületről, ahol a más (hozzáértő) felhasználók által elkészített
önálló vezérlő logikákat felhasználhatnák:

\begin{itemize}
  \item Egy adott billentyűre a felhasználó űrhajója automatikusan a
  legközelebbi ellenfél sebességvektorát felveszi, így könnyítve az amúgy nem
  túl intuitív navigációt.
  \item Az egér kurzorát mozgatva az űrhajó a megadott irányba áll (ez a művelet
  manuálisan különösen nehéz, azonban egy egyszerű szabályzóval könnyen
  megoldható).
\end{itemize}

\subsection{Több hajó vezérlése}

A játékosok alapvetően egy űrhajót irányítanak a játékban, mivel annak vezérlése
manuálisan történik, így egyszerre több űrhajót vezérelni egyszerűen túl komplex
feladat a játékosok számára. Mivel azonban a lehetőség adott a teljesen autonóm
vezérlő megírására, így érdemes lehet megvizsgálni azt a lehetőséget, hogy
egyszerre több űrhajót bízunk a játékosra.

Ezzel a funkcióval jóval komplexebb vezérlők implementálására is lehetőség
nyílik, hiszen az egyes űrhajóknak célszerű összedolgozniuk.

\subsection{Testreszabható játékelemek}

Az elkészült szoftver egy fajta játéküzemmódot, illetve egy fajta űrhajót
tartalmaz, azonban már a szoftver megírásakor figyelembe vettem, hogy ez a
későbbiekben változhat, így a későbbiekben könnyen lehet bele újabb típusú
űrhajókat, pl. könnyen navigálható, gyors, de gyenge páncélzatú űrhajót és
lomha, de jól páncélozott egységeket beépíteni.

A játék jellegét tekintve \emph{mindenki mindenki ellen} módban működik, érdemes
lehet a jövőben más módok bevezetését is kipróbálni (pl. csapatok küzdelme
egymás ellen).

\subsection{Egyszerűbb deployment folyamat}

Az üzemeltetés szempontjából a jelenlegi deploy folyamat egy komplett virtuális
gépet tekint a célkörnyezetnek, melynek a specifikációja nem ismert, így az
élesítés során számos ellenőrzés elvégzése szükséges (rendelkezésre állnak-e a
megfelelő npm csomagok, telepítve van-e a megfelelő adatbázis).

A folyamat egyszerűsíthető, ha virtuális gép helyett konténereket használunk,
pl. a napjainkban egyre népszerűbb Docker-t\cite{docker}. Az üzemeltetés ekkor
pusztán csak a megfelelő konténer futtatására terjed ki, magának a konténernek a
felkonfigurálása csak egyszer szükséges, így akkor elvégezhetjük a függőségek
telepítését. A megoldás másik előnye, hogy a szoftver izoláltan fut, így ha az
adott környezetben egy másik szoftvert is szeretnénk futtatni, ami pl. egy másik
adatbázis verziót igényel, akkor ezt minden további nélkül megtehetjük, hiszen
konténer technológia biztosítja, hogy a különböző verziók izoláltan fussanak.

\subsection{Automatikus skálázás}

Az alkalmazás által feltételezett infrastruktúra jelenleg statikus, azaz a játék
egy központi szerverként működik, nincs felkészítve kooperációra. Ez
nyilvánvalóan korlátozza a skálázási lehetőségeket, hiszen a rendelkezésre álló
erőforrások (CPU illetve memória kapacitás) csak korlátozottan bővíthetőek.

Architektúrális szempontból a lehetőség adott, hogy a játékot lebonyolító
szerverhez újabb szervereket adjunk. Ez a típusú skálázás horizontális, így
jóval rugalmasabb is. Ha pedig a rendszer erre felkészítjük, akár történhet
automatikusan is, azaz ha egy adott limitet elér a rendszer, automatikusan
elindul egy újabb példányban, ami képes újabb játékosokat fogadni.

Az implementáció bonyolultsága miatt azonban ez csak egy nagyon távoli cél, maga
a téma az összetettsége miatt pedig akár önálló szakdolgozat illetve diplomaterv
téma is lehetne.

\section{Összegzés}

Az elérhető webes technológiák összehasonlítása illetve kiértékelése után
kiválasztottam a megvalósításhoz szükséges megoldásokat, ezek segítségével
elkészítettem a feladatkiírásban definiált játékhoz a kliens illetve a szerver
alkalmazást. A tervezés során kiemelt figyelmet fordítottam a játék minél
realisztikusabb fizikai modelljére, illetve a kliens-szerver között történű
WebSocket kommunikáció optimalizálására. A játékba a külső vezérlés lehetőségét
is beépítettem, azaz megnyitottam a lehetőséget egyéni játékvezérlők írására is,
ezt egy példa vezérlő megírásával teszteltem. A játék kódját unit tesztekkel
fedtem le, illetve a Travis CI rendszert a fejlesztési folyamat szerves részévé
tettem. Az alkalmazás deploy igényeinek megvizsgálása után ezt éles környezetbe
is kihelyeztem, melyet részben automatizáltam, a későbbi élesítések
megkönnyítése érdekében. Végül összegeztem a tapasztalataimat, illetve néhány
izgalmasnak tűnő továbbfejlesztési lehetőséget is felvetettem.
