\chapter{Felhasznált technológiák}

\section{ECMAScript}

\subsection{A nyelv egyes verziói}

\subsection{Fordítási megoldások}

\subsubsection{Babel}

\subsubsection{Browserify}

\subsection{Erősen típusos alternatívák}

\subsubsection{Typescript}

\subsubsection{Flowtype}

\section{Node.js}

\subsection{Aszinkron működés}

\subsection{npm}

\section{koa}

\section{React}

\subsection{Szintaxis}

\subsection{Flux architektúra}

\section{Redux}

alternatívák: reflux, fluxxor, etc

\section{Webszerver keretrendszerek}

express, hapi, restify, koa

\subsection{express}

\subsection{koa}

\section{Adatbázis}

MySQL, NoSQL

\section{Kliens-szerver kommunikáció}

\subsection{REST}

\subsection{WebSocket}

\section{CSS stílusok}

CSS, SASS, LESS, Stylus

\subsection{LESS}

\subsection{Stylus}

\section{Grafika}

Canvas, WebGL, SVG

\subsection{Canvas}

\subsection{WebGL}

\section{Teszt keretrendszerek}

\section{WebWorker}

\section{Üzemeltetési megoldások}

Chef, Puppet, Anisble, Salt

\subsection{Ansible}
