\chapter*{Kivonat}\addcontentsline{toc}{chapter}{Kivonat}

A számítógépes játékokkal szemben számos követelményt támasztanak a
felhasználók, ilyen pl. az igényes grafika, valós idejű többjátékos üzemmód.
Ezeket az igényeket korábban a böngészőkön keresztül elhozni a felhasználókhoz
lehetetlen volt, azonban a modern, web alapú technológiák megjelenésével ez
megváltozott: a WebGL által a natív megoldásokkal összemérhető teljesítményű
grafikai megjelenítést vagyunk képesek megvalósítani, a Websocket technológia
által pedig képesek vagyunk a full-duplex valós idejű kommunikációra, mindehhez
csupán egy böngészőre van szükség.

A diplomatervem célja a jelenleg elérhető webes technológiák bemutatása, illetve
a segítségükkel egy játék elkészítése, a fejlesztés végigkövetése a tervezéstől
kezdve a tesztelésen át az üzemeltetéssel bezárólag. A megvalósítás egy Node.JS
alapú szerver, illetve egy böngészőben futtatható JavaScript kliens alkalmazásra
terjed ki. A szerver egy PostgreSQL adatbázis használ adattárolásra, a klienssel
történő kommunikáció WebSocketen protokollon keresztül történik, a felhasználói
felület pedig a Reacthoz hasonló virtual DOM alapú Inferno keretrendszerrel, és
WebGL technológia segítségével készült.

Az alkalmazás éles környezetben egy nginx reverse proxy mögött fut egy virtuális
szerveren, melyre az élesítési folyamatot az Ansible segítségével
automatizáltam.

\vfill

\chapter*{Abstract}\addcontentsline{toc}{chapter}{Abstract}

The users have a lot of demanding requirements for computer games, eg. nice
visual appearance, real-time multiplayer game mode.  These feaures were
basically impossible to implement in browsers, but with the development of
modern web technologies this is no longer the case: with WebGL we are able to
deliver 3D graphics that has the same performance as their native counterpart,
and with WebSockets we are able to implement full-duplex communication between
the client and the server, and all we need is just a regular browser.

The goal of my thesis is to present the currently available web technologies,
and to build a game based on these technologies. The implementation consists of
a Node.JS based server application and a browser-based JavaScript client
application. The server uses PostgreSQL to store data, the communication
protocol between the client and the server is based on WebSockets. The user
interface is built with the React-like virtual dom based Inferno framework, and
it also uses WebGL for graphics.

The application was deployed to a live environment, where the server runs behind
an nginx reverse proxy. The deployment process was automated with Ansible.

\vfill
